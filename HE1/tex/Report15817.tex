\documentclass[reprint, english,notitlepage,nofootinbib]{revtex4-1}  % defines the basic parameters of the document
% if you want a single-column, remove reprint

% allows special characters (including æøå)
\usepackage[utf8]{inputenc}
% \usepackage [norsk]{babel} %if you write norwegian
\usepackage[english]{babel}  %if you write english


%% note that you may need to download some of these packages manually, it depends on your setup.
%% I recommend downloading TeXMaker, because it includes a large library of the most common packages.

\usepackage{physics,amssymb}  % mathematical symbols (physics imports amsmath)
\usepackage{graphicx}         % include graphics such as plots
\usepackage{xcolor}           % set colors
\usepackage{hyperref}         % automagic cross-referencing (this is GODLIKE)
\usepackage{tikz}             % draw figures manually
\usepackage{listings}         % display code
\usepackage{subfigure}        % imports a lot of cool and useful figure commands
\usepackage{verbatim}
\usepackage{adjustbox}


% defines the color of hyperref objects
% Blending two colors:  blue!80!black  =  80% blue and 20% black
\hypersetup{ % this is just my personal choice, feel free to change things
    colorlinks,
    linkcolor={red!50!black},
    citecolor={blue!50!black},
    urlcolor={blue!80!black}}

%% Defines the style of the programming listing
%% This is actually my personal template, go ahead and change stuff if you want
\lstset{ %
	inputpath=,
	backgroundcolor=\color{white!88!black},
	basicstyle={\ttfamily\scriptsize},
	commentstyle=\color{magenta},
	language=Python,
	morekeywords={True,False},
	tabsize=4,
	stringstyle=\color{green!55!black},
	frame=single,
	keywordstyle=\color{blue},
	showstringspaces=false,
	columns=fullflexible,
	keepspaces=true}

\newcommand\numberthis{\addtocounter{equation}{1}\tag{\theequation}}
\newcommand{\ihat}{\boldsymbol{\hat{\textbf{\i}}}}
\newcommand{\jhat}{\boldsymbol{\hat{\textbf{\j}}}}
\newcommand{\khat}{\boldsymbol{\hat{\textbf{k}}}}
\newcommand{\oppg}[1]{\textbf{(#1)}}            % sub-task
\newcommand{\oppgg}[2]{\textbf{(#1) #2}}        % sub-task, 2 args
\newcommand{\svar}[1]{\underline{\underline{{#1}}}}
\newcommand{\vc}[1]{\mathbf{#1}}
\newcommand{\ddpart}[2]{\frac{\partial #1}{\partial #2}}
\newcommand{\ddtot}[2]{\frac{\mathrm d #1}{\mathrm d #2}}
\newcommand{\ddpartsq}[2]{\frac{\partial^2 #1}{\partial #2^2}}
\newcommand{\ddtotsq}[2]{\frac{\mathrm d^2 #1}{\mathrm d #2^2}}


\graphicspath{{../output/}} % search for figures in this dir


\begin{document}



\title{IN4200 Home Exam 1 - Counting shared nearest neighbors (C++)}
\date{\today}
\author{Candidate no.: 15817}


\newpage

% \begin{abstract}
% Abstract om du vil
% \end{abstract}
\maketitle                                % creates the title, author, date & abstract


\section{Ideas and algorithms}

\subsection{Reading the data files}

I will use the standard function \verb|fscanf| to process the data files. The function takes as argument the pointer to the \verb|FILE| object we want to read and moves through it until it reaches the character(s) or data type specified by the second argument. Matching decimal integers and assigning their value to variables already declared lets us read the numbers in the file. I have prefixed the calls to \verb|fscanf| with \verb|(void)| to tell the compiler that we are not interested in the return value of the function. This has no effect on the results, but at the very least it avoids some compiler warnings.


\subsection{\texttt{read\_graph\_from\_file1}} \label{sect:read1}

This function is rather simple. Let \(N\) denote the number of nodes in the network. We want to register the edges in an \(N \times N\) matrix of \verb|char| elements. I will use an underlying contiguous array of length \(N^2\) to do this. It is achieved by
\begin{verbatim}
  char *contig_array = new char[N*N];
  table2D = new char*[N];
  for (int i=0; i<*N; i++)
    table2D[i] = &contig_array[i*N];
\end{verbatim}
This can now be indexed by \verb|[row][column]| like a normal 2D-matrix. Letting \verb|table2D[i][j]| represent the link between nodes \verb|i| and \verb|j|, we will simply let 0 mean that there is no link between them and 1 mean that they are linked. Starting with an array of zeros, we then just set the relevant values to 1 as we read through the data file. The matrix is of course symmetric so that when we set \verb|table2D[i][j] = 1| we must also set \verb|table2D[j][i] = 1|.


\subsection{\texttt{read\_graph\_from\_file2}} \label{sect:read2}

Here, we want to store the information in the data file as two one-dimensional arrays using the CRS format. Again \(N\) denotes the number of nodes, and \(N_{\text{edges}}\) denotes the number of edges (connections between nodes). I will start by declaring an array of zeros, \verb|node_links|, to hold the number of other nodes connected to each node. I.e. if node 2 is connected to three other nodes, the value of \verb|node_links[2]| would be 3. This is easily counted while reading through the file. I will also store the information in the text file in the two arrays \verb|from_array| and \verb|to_array|, both of length \(N_{\text{edges}}\).

After reading through the file we want to store the information in two arrays: \verb|row_ptr| of length \(N+1\) and \verb|col_idx| of length \(2 N_{\text{edges}}\). The values of \verb|row_ptr| are easily calculated from the \verb|node_links| array. The array should start at 0. The second element should be the number of nodes connected to node 0, the third elemenent should be the sum of the second element and the number of nodes connected to node 1, the fourth element should be the sum of the third element and the number of nodes connected to node 2, etc. This is achieved by the following loop:
\begin{verbatim}
row_ptr[0] = 0;
for (int i=1; i<N+1; i++)
  row_ptr[i] = row_ptr[i-1] + node_links[i-1];
\end{verbatim}

After \verb|row_ptr| has been filled, I use the following code to fill \verb|col_idx| (which is currently full of zeros) with the correct values:
\begin{verbatim}
int from, to, col;
for (int i=0; i<N_edges; i++) {
  from = from_array[i];
  to = to_array[i];

  // insert the value showing that
  // "from" is connected to "to"
  col = row_ptr[from+1] - 1;
  while (col_idx[col] > to)
    col -= 1;
  for (int j=row_ptr[from]; j<col; j++)
    col_idx[j] = col_idx[j+1];
  col_idx[col] = to;

  // insert the value showing that
  // "to" is connected to "from"
  col = row_ptr[to+1] - 1;
  while (col_idx[col] > from)
    col -= 1;
  for (int j=row_ptr[to]; j<col; j++)
    col_idx[j] = col_idx[j+1];
  col_idx[col] = from;
}
\end{verbatim}
The integers \verb|from| and \verb|to| are the indices of two nodes that are connected. \verb|col| is the index for where in \verb|col_idx| to place the connection. To show that node \verb|from| is connected to node \verb|to| we must place the value \verb|to| in \verb|col_idx| in the area between indices \verb|row_ptr[from]| and \verb|row_ptr[from+1] - 1| (inclusive). Starting at the end of this interval, the first while loop moves the counter toward the front of the area if the places are already filled by larger numbers. After having moved past the larger entries, it settles on an index to place the number. The numbers ahead of this position are shifted to the left to make room. In this way, the CRS arrays are sorted after the algorithm is finished. The last code block above repeats the process with \verb|to| and \verb|from| switched (the matrix is symmetric).

It could have been better to start from the beginning of the intervals and move forward through them. The Facebook dataset is (atleast almost) sorted from lowest to highest such that the numbers would not be moved around as much if I traversed the interval in the other direction. However, there are a few small practical challenges to this if you want a sorted array that caused me to go for the strategy I have described.


\subsection{\texttt{create\_SNN\_graph1}}

Here, we want to calculate the number of shared nearest neighbors from the two-dimensional array of connections made by the function discussed in section \ref{sect:read1}. The strategy is simple: if nodes \verb|i| and \verb|j| are connected, that is if \verb|table2D[i][j]| (and \verb|table2D[j][i]|) is 1, then the corresponding SNN-value is given by the dot product between the two rows \verb|table2D[i]| and \verb|table2D[j]|. SNN's are the only elements in the arrays that will contribute to the dot product. For nodes \verb|k| that are connected to only one of \verb|i| and \verb|j| or neither of them we will have that \verb|table2D[i][k]*table2D[j][k]| is zero. Since the SNN-matrix is symmetric, we can manage by looping over only the upper or lower triangular part of the matrix as follows
\begin{verbatim}
for (int i=0; i<N; i++) {
  for (int j=i+1; j<N; j++) {
    if (table2D[i][j]) {
      // the two nodes are connected,
      // we need to calculate the SNN
      dot_product = 0;
      for (int k=0; k<N; k++)
        dot_product += table2D[i][k]*table2D[j][k];
      SNN_table[i][j] = dot_product;
      SNN_table[j][i] = dot_product;
    }
  }
}
\end{verbatim}

In order to parallelize this code I have simply instructed that the outermost for-loop be run in parallel. Since the size of the second for-loop decreases with \verb|i|, it is desirable to distribute the iterations between the threads such that iteration 0 goes to thread 0, iteration 1 to thread 1, etc. This is achieved by adding \verb|schedule(static, 1)| to the parallelization instruction. Thus the total computation time is somewhat evened out between the threads without the extra overhead of dynamic loop scheduling.


\subsection{\texttt{create\_SNN\_graph2}}

Here, we want to calculate the number of shared nearest neighbors from the one-dimensional CRS arrays of connections made by the function discussed in section \ref{sect:read2}. We want to store the results in an integer array \verb|SNN_val| of length \(2 N_{\text{edges}}\). Starting with this array full of zeros and exploiting that the arrays are sorted, I have implemented the following algorithm:
\begin{verbatim}
for (int i=0; i<N-1; i++) {
  start = row_ptr[i];
  end = row_ptr[i+1];
  // move past the pairs whose SNN
  // values have already been calculated
  while ((col_idx[start] < i)
          && (start < end))
    start++;

  // loop through the remaining neighbors
  for (int j=start; j<end; j++) {
    from_idx = row_ptr[i];
    to = col_idx[j];
    to_idx = row_ptr[to];

    mirror_idx = 0;
    matches = 0;

    // traverse the neighbors of both nodes
    // to check for SNNs
    while ((from_idx < end)
            && (to_idx < row_ptr[to+1])) {
      if (i == col_idx[to_idx]) {
        mirror_idx = to_idx;
        to_idx++;
      } else if (col_idx[from_idx] > col_idx[to_idx])
        to_idx++;
      else if (col_idx[from_idx] < col_idx[to_idx])
        from_idx++;
      else {
        // we have found an SNN
        matches++;
        from_idx++;
        to_idx++;
      }
    }
    // record the number of matches
    SNN_val[j] = matches;
    SNN_val[mirror_idx] = matches;
  }
}
\end{verbatim}
The outermost loop, over \verb|i|, goes through each of the nodes. For each node, the innermosts for-loop, over \verb|j|, goes through each of the neighbors to find how many shared nearest neighbors each pair has. As in the two-dimensional case, I enter the number of SNN's for each pair of nodes in both locations in the array the first time they are calculated. In this way, the for-loop over \verb|j| does not need to start at the first neighbor of \verb|i|. It only needs to loop through neighbors with indices higher than \verb|i| because the neighbors with indices lower than \verb|i| have already been calculated in a previous iteration of the for-loop over \verb|i|. The purpose of the first while-loop is to skip these entries.

After finding a pair of nodes for whom the SNN value has not yet been calculated, we have to traverse the neighbors of each of them to search for shared neighbors. This is done in the innermost while-loop. Here we can take advantage of the fact that the lists are ordered. The values of \verb|col_idx[from_idx]| and \verb|col_idx[to_idx]| increase as we increment \verb|from_idx| and \verb|to_idx|. Thus we can manage with only a single loop to traverse both lists, incrementing only the counter corresponding to the lowest value. If we for some set of indices have the same value we have found a shared neighbor. In that case we increment both indices and the counter for the number of SNN's.

My strategy for parallelizing this code was the same as in the previous section. The iterations of the outermost for-loop are run in parallel using static scheduling with step size 1 to even the computing time of the different threads as, in this example too, the loop over \verb|j| gets shorter as \verb|i| increases.


\subsection{\texttt{check\_node}}

In order to check whether a specific node can be in a cluster with the desired SNN limit \verb|tau| I first create an array \verb|cluster| of zeros of length \(N\) to hold the value of whether each node is in the cluster or not. Then we loop through each of the neighbors of the node given as input. For each of the neighbors we check if the number of SNN's between the pair is higher than or equal to \verb|tau|. If so, then the nodes can form a cluster, and the values in their locations in the array \verb|cluster| are set to 1.

For every neighboring node that can join the cluster, we also have to check \textit{their} neighboring nodes if \textit{they} can join the cluster, and so on until we reach a node who has no neighbors that can join the cluster and have not already done so. This is a recursive problem and I have implemented a second function called \verb|add_to_cluster| that takes in the index of a node that should be added to the cluster, adds it to the cluster and then goes through that node's neighbors to check if they too should join the cluster. The function calls itself every time it finds a new node which should be added to the cluster. The solution works well with datasets like the Facebook dataset and smaller, but I do not know the limitations of the implementation in terms of e.g. maximum recursion depth.



\onecolumngrid
\vspace{1cm} % some extra space
\newpage

% \bibliographystyle{plain}
% \bibliography{citations}  % bibtex: put references in a file called citations.bib


\end{document}
